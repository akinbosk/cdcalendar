%!TEX program=xelatex

% nobabel means you'll need to provide your own month names, week day headings,
% etc
\documentclass[12pt,nobabel,sundayweek]{cdcalendar}
%% and here we do some settings for a calendar in Chinese
%% (*not* the lunar calendar! Just localising the Gregorian calendar into
%% Chinese)
\usepackage{zh-mod}

%% Use Roboto and Roboto Slab
\usepackage[rm,light,osf,scale=0.94]{roboto}

% \title{Desktop Calendar (fits CD jewel case) with Chinese Localisation}

\usepackage{graphicx}
\usepackage{fontawesome}
\usepackage{wallpaper}
\usepackage[fleqn]{amsmath}
\usepackage{tikzlings,tikzducks}

\graphicspath{{img/}}

%% Define all event mark styles here
\tikzset{holiday/.style={rectangle,fill=orange!70}}
\tikzset{pink icon/.style={text=Pink,font=\large}}
\tikzset{blue icon/.style={text=SkyBlue,font=\large}}

\begin{document}

%%%%%%
% Cover
%%%%%%

\coverBgColor{RoyalBlue!40!black}
\coverImage[\color{gray!20!white}实际印刷出来的台历效果。较小尺寸的(\texttt{small} 选项,9\,cm $\times$ 9.5\,cm)适用3.5寸软盘盒(家里有这些古董的可以派上用场了)。较大尺寸的(默认设定,11.7\,cm $\times$ 13.65\,cm)则适用CD盒。另外还有\texttt{giant}选项,一页A4纸就一个月历。]
{actual-crop}

\coverTitle[
font=\fontsize{26pt}{30pt}\heiti,
text width=\linewidth,align=flush right,white]
{二零二四年历,再努力!}

\makeCover
\clearpage

% Remove this line if you feel the background pattern is too annoying
\TileWallPaper{.5\paperwidth}{.1\paperheight}{lightpaperfibers}

%%%%

% You may find the gap between illustrations and events too narrow;
% Use this length to incrase it
\setlength{\illusSkip}{1.5\ccwd}


%%%%%%
% Some settings for the monthly calendars
%%%%%%
\dayHeadingStyle{font=\sffamily\color{gray!90}}
\sundayColor{red}
\monthTitleStyle{font={\fontsize{42pt}{44pt}\bfseries\sffamily\fangsong}, red!50!RedViolet}
\eventStyle{\scriptsize\songti\color{black}}
\renewcommand\printeventname[1]{{\heiti\color{black}#1}}


%%%%%%
% January 2024
%%%%%%
\illustration
[Happy TikZ animals! This is an optional description about the illustrations.]
{0.9\linewidth}{tikzlings}

\begin{monthCalendar}{2024}{01}

%%% events must be given AFTER \begin{monthCalendar}
%%% Currently you must give events on the same page
%%% as the monthly calendar.

%% This is an one-day event
\event[mark style=holiday]{2024-01-01}{}{新年元旦}
%% This is a 5-day event
\event[mark style=blue icon,marker=\faBriefcase]{2024-01-30}{5}{ACME大会}
%% you could also write \event{2024-01-06}{2024-02-03}{ACME 大会}

\end{monthCalendar}

\clearpage

%%%%%%
% Feb 2024
%%%%%%

% Or you can put any stuff, really, with a caption if you want:
\setlength{\mathindent}{0pt}
\otherstuff[Fourier Transformation, one of the `math equations that changed the world'. \url{http://news.bitofnews.com/13-math-equations-that-changed-the-world/}]
{\linewidth}
{\huge\selectfont
\[%
  \hat{f}(\xi) = \int^{\infty}_{-\infty} f(x) e^{-2\pi ix\xi} \mathop{dx}
\]}

\begin{monthCalendar}{2024}{02}

%% Repeat the event if it spans two months
\event[mark style=blue icon,marker=\faBriefcase]{2024-01-30}{5}{ACME大会}
\event[mark style=pink icon,marker=\faBirthdayCake]{2024-02-07}{}{朋友生日}
\event{2024-02-24}{}{死线!!}

\end{monthCalendar}

\clearpage

%%% I... I can't search for more pics so will just show some cute animals
\otherstuff{\linewidth}{\tikz[scale=1.5]{\koala[crown]};}
\begin{monthCalendar}{2024}{03}
\end{monthCalendar}

\clearpage

\otherstuff{\linewidth}{\tikz[scale=1.5]{\duck[graduate,glasses]};}
\begin{monthCalendar}{2024}{04}
\end{monthCalendar}

\clearpage

\otherstuff{\linewidth}{\tikz[scale=1.5]{\mouse[cheese]};}
\begin{monthCalendar}{2024}{05}
\end{monthCalendar}

\clearpage

\otherstuff{\linewidth}{\tikz[scale=1.5]{\duck[crozier,strawhat=red!80!white,bowtie=red]};}
\begin{monthCalendar}{2024}{06}
\end{monthCalendar}

\clearpage

\otherstuff{\linewidth}{\tikz[scale=1.5]{\sloth[icecream]};}
\begin{monthCalendar}{2024}{07}
\end{monthCalendar}

\clearpage

\otherstuff{\linewidth}
  {\tikz[scale=1.5]{\duck[body=pink!50!white,bill=orange,
  unicorn=magenta!60!violet, longhair=magenta!60!violet]};}
\begin{monthCalendar}{2024}{08}
\end{monthCalendar}

\clearpage

\otherstuff{\linewidth}{\tikz[scale=1.5]{\coati[umbrella=blue!60!black,handbag]};}

\begin{monthCalendar}{2024}{09}
\end{monthCalendar}

\clearpage

\otherstuff{\linewidth}
  {\tikz[scale=1.5]{\duck[witch=black!50!gray,
      longhair=red!80!black,
      jacket=black!50!gray,
      magicwand]};}
\begin{monthCalendar}{2024}{10}
\end{monthCalendar}

\clearpage

\otherstuff{\linewidth}{\tikz[scale=1.5]{\duck[snowduck=white!90!gray,eye=white]};}
\begin{monthCalendar}{2024}{11}
\end{monthCalendar}

\clearpage

\otherstuff{\linewidth}{\tikz[scale=1.5]{\marmot[santa,wine]};}
\begin{monthCalendar}{2024}{12}
\end{monthCalendar}


\end{document}
