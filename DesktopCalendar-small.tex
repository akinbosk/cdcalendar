% \title{Desktop Calendar (fits 3.5" floppy disk jewel case)}

%%% The calendars are printed 2-up to fit CD jewel cases,
%%% or 4-up to fit 3" floppy disk jewel cases. OR, now a
%%% full-blown "giant" version that prints full-page!
%%%
%%% Localisation possible with languages supported by
%%% babel/translator/datetime2.
\documentclass[9pt,british,small]{cdcalendar}

%%% Use the sundayweek option for weeks to start on Sundays.
% \documentclass[9pt,british,small,sundayweek]{cdcalendar}


%% If using pdfLaTeX %%%%%%%%%%%%%%%%%%%%%%%%%%
\usepackage[T1]{fontenc}

\usepackage[rm,scale=0.94]{roboto}
\usepackage[notextcomp,notext]{stix2}
%% End pdfLaTeX-related font settings %%%%%%%%


% %% Compile with xelatex if using fontspec %%%%%%
% \usepackage{fontspec}
% \setmainfont{Gentium}
% \setsansfont[BoldItalicFont=Fira Sans Italic,BoldFont=Fira Sans]{Fira Sans Light}
% %% End xelatex-related font settings %%%%%%%%%%%

\usepackage{graphicx}
\usepackage{wallpaper}
\usepackage{fontawesome}
\graphicspath{{img/}}
\usepackage[fleqn]{amsmath}

\usepackage{tikzlings,tikzducks}

\tikzset{blue icon/.style={text=SkyBlue,font=\Large}}
\tikzset{pink icon/.style={text=Pink,font=\large}}
\tikzset{holiday/.style={rectangle,fill=orange!70}}

\begin{document}

%%%%%%
% Cover
%%%%%%
\coverBgColor{RoyalBlue!40!black}
\coverImage
  [\color{gray!80}(This is an optional description line for hte cover image.) Here are the actual printed calendars. The smaller calendar (9\,cm $\times$ 9.5\,cm) fits floppy disk jewel cases; while the bigger one (11.7\,cm $\times$ 13.65\,cm) fits CD jewel cases.]
  {actual-crop.jpg}
\coverTitle
  [font=\fontsize{20pt}{22pt}\sffamily\bfseries,text=white,text width=\linewidth,align=flush right]
  {2024 Calendar}

\makeCover
\clearpage
%%%%

%%%%%%
% Some settings for the monthly calendars
%%%%%%
% \captionStyle{font=\sffamily\itshape\tiny}
\dayHeadingStyle{font=\sffamily,text=gray!90}
\sundayColor{red!80!black}
\monthTitleStyle{font={\fontsize{26pt}{28pt}\bfseries\sffamily\itshape}, text=red!50!RedViolet}
\eventStyle{\scriptsize\sffamily}

% Remove this line if you feel the background pattern is too annoying
\TileWallPaper{.5\paperwidth}{.5\paperheight}{lightpaperfibers}

% You may find the gap between illustrations and events too narrow;
% Use this length to increase it
\setlength{\illusSkip}{1em}


%%%%%%
% January 2020
%%%%%%
\illustration
[Happy TikZ animals! This is an optional description about the illustrations.]
{0.9\linewidth}{tikzlings}

\begin{monthCalendar}{2024}{01}

%%% events must be given AFTER \begin{monthCalendar}
%%% Currently you must give events on the same page
%%% as the monthly calendar.

%% This is an one-day event
\event[mark style=holiday]{2024-01-01}{}{New Year's Day}
%% This is a 5-day event
\event[mark style=blue icon,marker=\faBriefcase]{2024-01-30}{5}{ACME Conference}
%% you could also write \event{2024-01-06}{2024-02-03}{ACME Conference}

\end{monthCalendar}

\clearpage

%%%%%%
% Feb 2020
%%%%%%

% Or you can put any stuff, really, with a caption if you want:
\setlength{\mathindent}{0pt}
\otherstuff[Fourier Transformation, one of the `math equations that changed the world'. \url{http://news.bitofnews.com/13-math-equations-that-changed-the-world/}]
{\linewidth}
{\huge\selectfont
\[%
  \hat{f}(\xi) = \int^{\infty}_{-\infty} f(x) e^{-2\pi ix\xi} \mathop{dx}
\]}

\begin{monthCalendar}{2024}{02}

%% Repeat the event if it spans two months
\event[mark style=blue icon,marker=\faBriefcase]{2024-01-30}{5}{ACME Conference}
\event[mark style=pink icon,marker=\faBirthdayCake]{2024-02-07}{}{Someone's birthday}
\event{2024-02-24}{}{Grant proposal deadline!!}

\end{monthCalendar}

\clearpage

%%% I... I was too tired search for more pics so will just show some cute animals
\otherstuff{\linewidth}{\tikz[scale=1.5]{\koala[crown]};}
\begin{monthCalendar}{2024}{03}
\end{monthCalendar}

\clearpage

\otherstuff{\linewidth}{\tikz[scale=1.5]{\duck[graduate,glasses]};}
\begin{monthCalendar}{2024}{04}
\end{monthCalendar}

\clearpage

\otherstuff{\linewidth}{\tikz[scale=1.5]{\mouse[cheese]};}
\begin{monthCalendar}{2024}{05}
\end{monthCalendar}

\clearpage

\otherstuff{\linewidth}{\tikz[scale=1.5]{\duck[crozier,strawhat=red!80!white,bowtie=red]};}
\begin{monthCalendar}{2024}{06}
\end{monthCalendar}

\clearpage

\otherstuff{\linewidth}{\tikz[scale=1.5]{\sloth[icecream]};}
\begin{monthCalendar}{2024}{07}
\end{monthCalendar}

\clearpage

\otherstuff{\linewidth}
  {\tikz[scale=1.5]{\duck[body=pink!50!white,bill=orange,
  unicorn=magenta!60!violet, longhair=magenta!60!violet]};}
\begin{monthCalendar}{2024}{08}
\end{monthCalendar}

\clearpage

\otherstuff{\linewidth}{\tikz[scale=1.5]{\coati[umbrella=blue!60!black,handbag]};}

\begin{monthCalendar}{2024}{09}
\end{monthCalendar}

\clearpage

\otherstuff{\linewidth}
  {\tikz[scale=1.5]{\duck[witch=black!50!gray,
      longhair=red!80!black,
      jacket=black!50!gray,
      magicwand]};}
\begin{monthCalendar}{2024}{10}
\end{monthCalendar}

\clearpage

\otherstuff{\linewidth}{\tikz[scale=1.5]{\duck[snowduck=white!90!gray,eye=white]};}
\begin{monthCalendar}{2024}{11}
\end{monthCalendar}

\clearpage

\otherstuff{\linewidth}{\tikz[scale=1.5]{\marmot[santa,wine]};}
\begin{monthCalendar}{2024}{12}
\end{monthCalendar}


\end{document}
